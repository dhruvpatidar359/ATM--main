
\documentclass[11pt]{article}
\usepackage{graphicx}
\usepackage{listings}
\graphicspath{ {.C:/Users/dhruv/Desktop/ATM--main/} }
\begin{document}
Name:$\textbf{Dhruv Patidar}$\\

Enrollment No:$\textbf{0801CS211036}$

$$\textbf{Report on mini-project ( ATM )}$$

$$Characterstics \hspace{5pt} of\hspace{5pt} MiniProject$$
Starting Date/Time : 6 Nov ,2022 StartTime:10 AM Morning\\
End Data/Time : 15 Nov ,2022 EndTime:11 PM \\
Total Time Required :15 hours \\
Total Line of Code : 600+ lines\\
Number of Functions : 13\\\\
$\textbf{Objectives of Project} :$ \\\\
1$>$ This project is intended to have a better functionality in $\textbf{ATM}$ system , as this project also shows the graph for the withdraw and deposit per month and also display them separately\\\\
2$>$ This project/software have audio facility in it , it also tells the instruction to the customer , like " Enter name ..." and also tells that what wrong they have entered therefore increasing the communication gap between the machines and human\\\\
$\textbf{Function Descriptions} :$\\\\
1$>$ $\_\_main\_\_$ :\\\\
This is the driver function of the code/program , it simply calls the menu() function.\\\\
2$>$ menu():\\\\
This function ask , whether you want to Login/SignUp/Exit the system.\\\\
3$>$ aadharCheck.cpp:\\\\
This function or file checks whether the aadhar number entered is correct or not.
4$>$ passCheck():\\\\
This function checks whether the password entered by the user secured or not , or it is easy to crack.\\\\
5$>$ checkEmail.cpp:\\\\
This function or file checks whether the mail entered is correct or not\\\\
6$>$ ver():\\\\
This function handles the vercode.txt handling , which is used as a security OTP for the person that is trying to withdraw/deposit in the account.\\\\
7$>$ login():\\\\
This function allows the user to login in into the account , by checking whether the person with the given username and password exist or not.\\\\
8$>$ afterlogin():\\\\
This function help the user to do the functionalities like deposit , withdraw , see transaction history , see account details and graphs.\\\\
9$>$ choiceMainMenu():\\\\
This function helps in making choice in afterlogin function , it actually identify whether the choice entered is valid or not in afterlogin choice system.\\\\
10$>$ graphChoiceChecker():\\\\
This function is used to perform choice function in graphmenu function.\\\\
11$>$ graphMenu():\\\\
This functions is responsible for showing the graph to the user , that makes easier for the user to understand about his transaction history.\\\\
12$>$ choiceAfterLogin.cpp :\\\\
This is the choice selector file for afterLogin function in python
13$>$ choiceLogin.cpp:\\\\
This is the choice selector for login menu

$$ PROFILING $$
\includegraphics[scale=0.2]{profile1.png}\\
                               \\\\
\includegraphics[scale=0.2]{profile2.png}\\
                               \\\\
\includegraphics[scale=0.2]{profile3.png}\\
                               \\\\
\includegraphics[scale=0.2]{profile4.png}\\
                               \\\\
\includegraphics[scale=0.2]{profile5.png}\\
\\\\

$$DEBUGGING \hspace{5pt} USED$$\\
First Debug -- Error--\\
\includegraphics[scale=0.2]{error1.png}\\
                               \\\\
--Caught the error using debugger--(pdb in python)\\
                               \\\\
\includegraphics[scale=0.2]{error1debug.png}\\
                               \\\\
--Corrected the code--\\
                               \\\\
\includegraphics[scale=0.2]{error1sol.png}\\
                               \\\\
Second Debug --Error--\\
                               \\\\
\includegraphics[scale=0.2]{error2.png}\\
                               \\\\
--Caught the error using debugger-- (pdb in python)\\
                               \\\\
\includegraphics[scale=0.2]{error2debug.png}\\
                               \\\\
--Corrected the code--\\
                               \\\\
\includegraphics[scale=0.2]{error2sol.png}\\
                               \\\\





\lstinputlisting[language=Python]{atm.py}
$$\LARGE{C++ \hspace{5pt}code \hspace{5pt}used \hspace{5pt}in \hspace{5pt}this\hspace{5pt} project}$$\\
This is the choiceLogin.cpp file\\
\lstinputlisting[language=C++]{choiceLogin.cpp}
This is the aadharCheck.cpp file\\
\lstinputlisting[language=C++]{aadharCheck.cpp}
This is the checkEmail.cpp file\\
\lstinputlisting[language=C++]{checkEmail.cpp}
This is the choiceAfterLogin file\\
\lstinputlisting[language=C++]{choiceAfterLogin.cpp}


 $$OUTPUT \hspace{5pt} OF \hspace{5pt} THE \hspace{5pt} CODE$$ \\
\begin{center}
\large{This is the sign up part of the ATM system}\\
\includegraphics[scale=0.2]{img1.png}\\
\includegraphics[scale=0.2]{img2.png}\\
\large{This is the second menu where you can do other functionalities}\\
\includegraphics[scale=0.2]{img3.png}\\
\includegraphics[scale=0.2]{img4.png}\\
\includegraphics[scale=0.2]{img5.png}\\
\includegraphics[scale=0.2]{img6.png}\\
\includegraphics[scale=0.2]{img7.png}\\
\includegraphics[scale=0.2]{img8.png}\\
\includegraphics[scale=0.2]{img9.png}\\
\includegraphics[scale=0.2]{img10.png}\\
\includegraphics[scale=0.2]{img11.png}\\
\includegraphics[scale=0.2]{img12.png}\\
\end{center}



\end{document}
